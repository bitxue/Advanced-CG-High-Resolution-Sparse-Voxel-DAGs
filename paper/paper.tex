% Copyright 2004 by Till Tantau <tantau@users.sourceforge.net>.
%
% In principle, this file can be redistributed and/or modified under
% the terms of the GNU Public License, version 2.
%
% However, this file is supposed to be a template to be modified
% for your own needs. For this reason, if you use this file as a
% template and not specifically distribute it as part of a another
% package/program, I grant the extra permission to freely copy and
% modify this file as you see fit and even to delete this copyright
% notice. 

\documentclass{beamer}
\usepackage{graphicx}

% There are many different themes available for Beamer. A comprehensive
% list with examples is given here:
% http://deic.uab.es/~iblanes/beamer_gallery/index_by_theme.html
% You can uncomment the themes below if you would like to use a different
% one:
%\usetheme{AnnArbor}
%\usetheme{Antibes}
%\usetheme{Bergen}
%\usetheme{Berkeley}
%\usetheme{Berlin}
%\usetheme{Boadilla}
%\usetheme{boxes}
%\usetheme{CambridgeUS}
%\usetheme{Copenhagen}
%\usetheme{Darmstadt}
%\usetheme{default}
%\usetheme{Frankfurt}
%\usetheme{Goettingen}
%\usetheme{Hannover}
%\usetheme{Ilmenau}
%\usetheme{JuanLesPins}
%\usetheme{Luebeck}
\usetheme{Madrid}
%\usetheme{Malmoe}
%\usetheme{Marburg}
%\usetheme{Montpellier}
%\usetheme{PaloAlto}
%\usetheme{Pittsburgh}
%\usetheme{Rochester}
%\usetheme{Singapore}
%\usetheme{Szeged}
%\usetheme{Warsaw}

\title{High Resolution Sparse Voxel DAGs}


\author{Prateek Chandan \and Anurag Shirolkar}
% - Give the names in the same order as the appear in the paper.
% - Use the \inst{?} command only if the authors have different
%   affiliation.

\institute[IIT Bombay] % (optional, but mostly needed)
{
  \inst{}%
  Department of Computer Science\\
  IIT Bombay
}
% - Use the \inst command only if there are several affiliations.
% - Keep it simple, no one is interested in your street address.

\date{CS 775}
% - Either use conference name or its abbreviation.
% - Not really informative to the audience, more for people (including
%   yourself) who are reading the slides online

\subject{Advanced Computer Graphics}
% This is only inserted into the PDF information catalog. Can be left
% out. 

% If you have a file called "university-logo-filename.xxx", where xxx
% is a graphic format that can be processed by latex or pdflatex,
% resp., then you can add a logo as follows:

% \pgfdeclareimage[height=0.5cm]{university-logo}{university-logo-filename}
% \logo{\pgfuseimage{university-logo}}

% Delete this, if you do not want the table of contents to pop up at
% the beginning of each subsection:
\AtBeginSubsection[]
{
  \begin{frame}<beamer>{Outline}
    \tableofcontents[currentsection,currentsubsection]
  \end{frame}
}

% Let's get started
\begin{document}

\begin{frame}
  \titlepage
\end{frame}

\begin{frame}{Outline}
  \tableofcontents
  % You might wish to add the option [pausesections]
\end{frame}

% Section and subsections will appear in the presentation overview
% and table of contents.
\section{Abstract}

	\begin{frame}{Abstract}
	\begin{itemize}
		\item {
		A binary voxel grid can be represented more efficiently by using a Directed Acyclic Graph(DAG) than Sparse Voxel Octree(SVO)
		}
		\item {
			Along with efficient encoding of empty regions , DAG allows efficient encoding of identical regions as nodes are allowed to share pointers
		}
		\item {
			This paper presents a bottom-up algorithm to reduce a SVO to DAG
		}
		\item {
			The memory consumption of DAG is significantly smaller and it would fit in memory where SVO won't
		}
		\item {
			DAG requires no decompression and can be traversed easily
		}
		\item {
			This paper demonstrates this by ray tracing hard and soft shadows , ambient occlusion , and primary rays in extremly high resolution DAG at speed at par or even faster than voxel and triangle GPU ray tracing.
		}
	\end{itemize}
\end{frame}

\section{Introduction}

% You can reveal the parts of a slide one at a time
% with the \pause command:
\begin{frame}{Introduction}
  \begin{itemize}
  \item {
    There is increased interest in evaluating secondary rays for effects as reflection, shadows and indirect illumination. Due to the incoherence, a scondary scene representation is required with strict memory budget
    
  }
  \item {   
    This acceleration structure has to be resident on RAM for effiecient access.
  }
  \item{
  However this is problematic when triangle meshes are augmented with displacement maps as they become infeasibly large to millions of polygons
  }
  \item {
  	Recently Sparse Voxel Octree(SVO) has been used as a secondary scene representation, since they can be efficiently ray traced in real time.
  }
  \item {
  	At high resolution, SVO is still too much memory expensive, therefore it's application has been limited }
  \end{itemize}
\end{frame}

\begin{frame}{Introduction}
  \begin{itemize}
  \item {
  		The data structure \textbf{DAG} allows for extremly high resolution enabling us to improve image quality, decrease discretization artifacts, explore high frequency effects like sharp shadows in large scenes.
    }
 
  \end{itemize}
  \includegraphics[scale=0.36]{fig1}{C Figure 1: The EPIC CITADEL scene voxelized to a $128K^3$ resolution and stored as a Sparse Voxel DAG.}
\end{frame}

\begin{frame}{Introduction}
  \begin{itemize}
  \item {
  		The paper presents and efficient technique for reducing the size of SVO by merging common subtrees and transforming to a DAG.
    }
    \item{
    	This can dramatically reduce memory demands in real world scenes and scales even better with increased resolutions.
    }
    \item{
    	This is based on assumption that the original DAG contains a large amount of redundant subtrees
    }
    \item{
    	This paper implements and evaluates algorithms for calculating hard and soft shadows and ambitent occlusion by Ray tracing in DAG 
    }
 
  \end{itemize}
 
\end{frame}

\section{Previous Work}
\subsection{Sparse Voxel Octrees}
\begin{frame}{Sparse Voxel Octrees}
\begin{itemize}
	\item An octree is a spatial data structure, which subdivides each node into 8 equally large subnodes
	\item Sparse Voxel Octree has most of the nodes empty and its base primitive is a voxel.
	\item \textit{Efficient Sparse Voxel Octree} - avoid building
the octree to its maximal depth by providing each cubical voxel with
a contour
\end{itemize}
\end{frame}

\subsection{Common subtree merging}
\begin{frame}{Common subtree merging}
\begin{itemize}
	\item Work by Webber and Dillencourt [1989], where quadtrees  are compressed by merging common subtree
	\item Shows a significant decrease (up to 10×) in the
number of nodes required to represent a 512 × 512 binary image
	\item Parker et al. [2003] extends this to three dimensions
\end{itemize}
\includegraphics[scale=0.3]{images/CSM.jpg}{ Figure 2:}
\end{frame}

\subsection{Shadows}
\begin{frame}{Shadows}
\begin{itemize}
	\item \textit{Image based methods} - occluders are somehow recorded as seen from light source in an image, which is later used to determine light visibility while rendering.
	\item \textit{geometry based methods} - idea of generating shadow volumes that enclose shadowed space. Extended to Soft Shadow Volumes in case of area lights.
	\item Due to recent advances in real-time ray tracing performance, casting a few shadow rays per pixel is now within reach for interactive applications
\end{itemize}
\end{frame}

\subsection{Ambient Occlusion}
\begin{frame}{Ambient Occlusion}
\begin{itemize}
	\item Technique to determine exposure of each point to ambient light.
	\item \textit{Screen Space Ambient Occlusion} - Scaled down version of AO where only scene depth buffer is used to determine the occlusion.
	\item In an alternative approach a volume is generated per primitive with occlusion being computed at runtime
	\item \textit{Fast Ray-Cast Ambient Occlusion} - occlusion volume is limited by a hemisphere around the reciver point.
\end{itemize}
\end{frame}

\section{Reducing a Sparse Voxel Tree to a DAG}
\begin{frame}{Reducing a Sparse Voxel Tree to a DAG}
	\includegraphics[scale=0.37]{fig2}{ Figure 2:Reducing a sparse voxel tree, illustrated using a binary tree, instead of an octree, for clarity.}
	\begin{enumerate}[a.]
		\item{The original tree.}
		\item{Non-unique leaves are reduced.}
		\item{Now, there are non-unique nodes in the level above the leaves. These are reduced, creating non-unique nodes in the level above this.}
		\item{The process proceeds until we obtain our final directed acyclic graph.}
	\end{enumerate}
\end{frame}

\begin{frame}{Terminology}
	\begin{itemize}
		\item{
		Consider an $N^3$ voxel grid as a scene representation where each cell can be represented by a bit : \textbf{0} if empty and \textbf{1} if it contains geometry
		}
		\item{
			A sparse voxel octree recursivley divided $L$ time gives us a hierarchical representation of and $N^3$ voxel grid, where $N = 2^L$ and L is called as \textit{max level}.
		}
		\item{
			A node is implemented with a \textit{childmask}, a bitmask of 8 bits where bit $i$ tells us if bit $i$  contains a geometry, and a pointer to first of non-empty children.
		}
		\item{
			Child nodes are stored consecutively in memory	
		}
		\item{
			The unique traversal \textit{path}, from root to a specific node in SVO, defines the voxel without storing spatial information explicitly in the node 
		}
	\end{itemize}
\end{frame}


\begin{frame}{The sparse voxel DAG}
	\begin{itemize}
		\item{
		To Transform an SVO to a DAG we will use a bottom up approach.
		}
		\item{
		The leaf nodes are defined by uniquely by childmask of 8-bit, so there can be atmost $2^8=256$ unique leaf nodes}
		\item{
		Steps of Reduction:
			\begin{enumerate}
				\item{Merge identical leaves with same bitmask}
				\item{Proceed to level above and mege nodes with \textit{Identical childmask} and \textit{Identical pointers}}
				\item{Iteratively perform this step by merging larger and larger subtrees}
			\end{enumerate}
		}
		\item{The smallest possible DAG is guranteed to be found in L iterations}
		\item{
		This is also applicable for non-uniform octrees and partially reduced DAG's
		}
		\item{
		This allows us to conveniently build DAGs from	SVOs too large to reside in RAM or on disk.
		}
		
	\end{itemize}
\end{frame}

\begin{frame}{Implementation Details - DAG Construction}
\begin{itemize}
\item{While building a dag of level $L>10$ we first construct top L-10 levels by triangle/cube intersection in a top down fashion}
\item{For each leaf we construct the subtree till mazimum depth}
\item{The subtree is reduced to DAG and then we proceed to next leaf}
\item{After all sub-DAG's get have been constructed, we have top-level SVO with many DAG}
\item{We apply reduction one last time to produce one final DAG}
\item{The result is same as if we processed complete SVO but the memory consumption while construction is considerably low}
\end{itemize}
\end{frame}

\begin{frame}{Implementation Details - Merging Nodes}
\begin{itemize}
\item{We sort the nodes of a level using a \textit{256-bit} $($8 child * 32-bit pointer$)$ key and pointer as value}
\item{Identical nodes become adjacent and can be compacted to a single nodes}
\item{We also maintain a table of inderection, for each node we know the pointer to compacted node}
\item{To imporve performance we store subtrees of resolution $4^3$ of last two rows into a 64-bit integer}
\end{itemize}
\end{frame}

\section{Ray Tracing a Sparse Voxel DAG}

\begin{frame}{Ray Tracing a Sparse Voxel DAG}
	\begin{itemize}
	\item{The paper implements a GPU-based Raytracer in CUDA that efficiently traverses the scene representation}
	\item{The raytracer is primarily used for visibility queries to evaluate hard shadows, soft shadows, and ambient occlusion
	}
	\item{Ambient occlusion is usually approximated by casting a number
of rays over the hemisphere and averaging the occlusion of each
ray}
	
	\end{itemize}
\end{frame}

\begin{frame}{Evaluation}
	\includegraphics[scale=0.26]{images/scenes.png}{\\Figure 3:Scenes used in the experiments.}
\end{frame}

\section{Evaluation}
\subsection{Reduction Speed}
\begin{frame}{Reduction Speed}
	\includegraphics[scale=0.23]{images/red-speed.png}{\\Figure 3:Time taken to reduce the SVOs to DAGs. The first row is the total time and the second row is the part of that spent on sorting.}
	\begin{itemize}
		\item Crassin et al. [2012] SVO - CRYSPONZA at resolution $512^3$ - 7.34ms
		\item Laine and Karras [2010a] ESVO - HAIRBALL at resolution $1K^3$ -628s.
	\end{itemize}
	
	
\end{frame}
\subsection{Memory Consumption}
\begin{frame}{Memory Consumption}
	\includegraphics[scale=0.27]{images/mem-consumption.png}{\\Figure 4:Comparison of the sparse voxel DAG, ESVO and SVO. Cases where the DAG has the smallest memory consumption are highlighted in green}
\end{frame}

\subsection{Traversal Speed}
\begin{frame}{Traversal speed - Primary Rays}
\includegraphics[scale=0.35]{images/primary.png}{}
\end{frame}

\begin{frame}{Traversal speed - Shadow Rays}
\includegraphics[scale=0.35]{images/shadow.png}{}
\end{frame}

\begin{frame}{Traversal speed - Ambient Occlusion}
\includegraphics[scale=0.35]{images/AO.png}{}
\end{frame}

\section{Conclusion}
\begin{frame}{Conclusion}
We have shown
\begin{itemize}
	\item that sparse voxel DAGs, an evolution of sparse
voxel octrees, allow for an efficient encoding of identical regions of
space
	\item An algorithm to construct smallest possible DAG for the given Sparse Voxel Octree
	\item that this construction is possible without having to store the entire SVO in memory
	\item increase in node size is quickly amortized by reduction in node count.
	\item Despite smaller memory footprint, our Data structure can be efficiently ray traced.
\end{itemize}

\end{frame}
\section{Future Work}
\begin{frame}{Future Work}
\begin{itemize}
  	\item The memory layout of the DAG used in this paper could potentially
be improved
	\item We could add material representation residing in a separate tree or DAG which would be used on finding intersection of ray and the geometry.
	\item Future work could introduce cycles in the graph to fake unlimited resolution
\end{itemize}
\end{frame}



\begin{frame}
\begin{center}
\Huge{Thanks}\\


\LARGE{Questions?}
\end{center}
\end{frame}

\end{document}


